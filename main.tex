\documentclass[12pt,oneside]{book}
\usepackage[utf8]{inputenc}
\usepackage[english]{babel}

% header and footer support
\usepackage{fancyhdr}
\pagestyle{fancy}
\fancyhf{}
\fancyhead[L]{Alistair O'Brien}
\fancyhead[R]{NST Workbook}
\fancyfoot[C]{\leftmark}
\fancyfoot[R]{\thepage}
\renewcommand{\headrulewidth}{1pt}
\renewcommand{\footrulewidth}{1pt}

% chapter titles
\usepackage{titlesec}
\titleformat{\chapter}
{\normalfont\huge\bfseries}{\thechapter}{20pt}{}
\titlespacing*{\chapter} {0pt}{20pt}{40pt}

% figure support
\usepackage{import}
\usepackage{xifthen}
\pdfminorversion=7
\usepackage{pdfpages}
\usepackage{transparent}
\newcommand{\incfig}[1]{%
    \def\svgwidth{\columnwidth}
    \import{./figures/}{#1.pdf_tex}
}


\pdfsuppresswarningpagegroup=1

% math support
\usepackage{amsmath, amssymb, amsthm}
\DeclareMathOperator{\sech}{sech}
\DeclareMathOperator{\cosech}{cosech}

\DeclareMathOperator{\arcosh}{arcosh}
\DeclareMathOperator{\arsinh}{arsinh}
\DeclareMathOperator{\artanh}{artanh}
\DeclareMathOperator{\arsech}{arsech}
\DeclareMathOperator{\arcosech}{arcosech}
\DeclareMathOperator{\arccosec}{arccosec}
\DeclareMathOperator{\arcsec}{arcsec}
\DeclareMathOperator{\arccot}{arccot}
\DeclareMathOperator{\arcoth}{arcoth} 
\DeclareMathOperator{\dom}{dom} 
\DeclareMathOperator{\rng}{rng} 
\DeclareMathOperator{\cosec}{cosec} 


\begin{document}
    \thispagestyle{empty}
    \begin{center}
        {\LARGE Queens' College Cambridge}\\[1.5cm]
        \linespread{1.2}\huge {\bfseries NST Workbook}\\[1.5cm]
        \linespread{1}
        \includegraphics[width=5cm]{images/Queens.png}\\[1cm]
        {\Large Alistair O'Brien}\\[1cm]
        {\Large Department of Computer Science}\\[1.5cm]
        {\Large \today}
    \end{center}

    \newpage

    \begin{enumerate}
	\item \textbf{A1} $x^{-10}$    
        \item \textbf{A2} \begin{enumerate}
            \item We have $x ^2 - 1 = (x - 1)(x + 1)$.
            \item We have $a^2 - 4ab + 4b^2 = (a + 2b)(a - 2b)$.
            \item We have $x ^3 - 1 = (x - 1)(x^2 + x + 1)$.
        \end{enumerate}
        \item \textbf{A3} \begin{enumerate}
            \item Given that $x^2 - 5x + 6 = (x-2)(x-3) = 0$ then it follows that the solutions are $x_{1,2} = 2, 3$.
            \item Given that $x^2 + 2x = x (x+2) = 0$ then it follows that the solutions are $x_{1,2} = 0, -2$.
            \item We have the quadratic equation $x^2 - x - 1 = 0$. Given that we cannot find any roots by inspection, let us applying the quadratic formula, which gives us the following solutions \[
                x_{1,2} = \frac{-(-1) \pm \sqrt{1 - 4 \times 1 \times (-1) } }{2} = \frac{1 \pm \sqrt{5}}{2}
            .\]  
            \item We have the quartic equation $x^4 - 3x^2 + 2 = 0$. Using the substitution $u = x^2$, we have the quadratic equation $u^2 - 3u + 2 = 0$. Factorising this yields \[
                u^2 - 3u + 2 = (u - 2) (u - 1) = 0
            .\]
            Hence the roots of the above equation are $u_{1,2} = 1, 2$. Since $u = x^2$, then we have the following solutions to the original equation \[
                x_{1,\ldots,4} = \pm 1, \pm \sqrt{2}
            .\]     
        \end{enumerate}
        \item \textbf{A4}. Recall that the vertex from of univariate quadratic function is given by \[
            f(x) = ax^2 + bx + c = a \left( x - \frac{b}{2a} \right)^2 + \left( c - \frac{b^2}{4a} \right)
        .\] where $a, b, c \in \mathbb{R}$. Also note that $\forall x \in \mathbb{R}, x^2 \geq 0$, so it follows that \[
            \left( x - \frac{b}{2a} \right)^2 \geq 0
        .\] for all $x \in \mathbb{R}$. So the minima (or maxima) of $f$ occurs when \[
            \left( x - \frac{b}{2a} \right)^2 = 0 \iff x = \frac{b}{2a}
        .\] And so any univariate has the turning point \[
            \left( \frac{b}{2a},  c - \frac{b^2}{4a}\right)
        .\] 
        
            \begin{enumerate}
            \item So writing the quadratic function in vertex form gives us \[
                f(x) = x^2 - 2x + 6 = (x - 1)^2 + 5
            .\]
            Using the argument above, the minimum value of $f$ is $5$ when $x = 1$.
            \item Similarly \[
                g(x) = x^4 + 2x^2 + 2 = (x^2 + 1)^2 + 1
            .\] Hence the minimum value of $g$ is $1$ when $x = \pm 1$.
            \item Given the minimum of (a) occurs at $x = 1$ it follows that the shape of (a) in the domain $2 \leq x \leq 3$ is concave, thus the minimum value occurs at the boundary $x = 2$. Substituting this in gives us $f(2) = (2 - 1)^2 + 5 = 6$.
        \end{enumerate}
        \item \textbf{A5} \begin{enumerate}
            \item We have \begin{align*}
                x^2 - 3x &< 4 \\
                \iff x^2 - 3x - 4 &< 0 \\
                \iff (x + 1) (x - 4) &< 0.
            \end{align*}
            Hence $x^2 - 3 < 4 \iff x \in (-1, 4)$.
            \item We have \begin{align*}
                y^3 &< 2y^2 + 3y \\
                \iff y^3 - 2y^2 - 3y &< 0 \\
                \iff y(y - 3)(y + 1) &< 0.
            \end{align*}
            Hence $y^3 < 2y^2 + 3y \iff x \in (- \infty, -1) \cup (0,3) $.
        \end{enumerate}
        \item \textbf{A6} \begin{enumerate}
            \item We note that $(x+4)$ is a factor of  $f(x) = x^3 + 5x^2 - 2x -24$ since $x = -4$ is a root of $f$, so by the factor theorem $(x + 4)$ is a factor of $f$. So \[
                f(x) = (x+4)g(x) 
            .\] where $g$ is some univariate polynomial of degree 2. So using polynomial division, we have \begin{align*}
                g(x) &= \frac{x^3 + 5x^2 - 2x - 24}{x + 4} \\
                &= \frac{x^2 (x + 4) + x^2 - 2x - 24}{x+4} \\
                &= x^2 + \frac{x^2 -2 x - 24}{x+4} \\
                &= x^2 + \frac{x(x + 4) -6x - 24}{x+ 4} \\
                &= x^2 + x - 6\frac{x+ 4}{x+4} \\
                &= x^2 + x - 6 \\
                &= (x - 2) (x + 3).
            \end{align*}
            Hence the fully factorised form of $f$ is \[
                f(x) = (x + 4)(x + 3)(x - 2)
            .\] 
            \item By inspection we note that $t = 1$ is a root of $f(t) = t^3 - 7t + 6$. So by the factor theorem we have \[
                f(t) = (t - 1) g(t)
            ,\] where $g(t)$ is some univariate polynomial of degree 2. Intuitively we can see that $g$ has the following form \[
                g(t) = t^2 + at - 6
            .\]  Given that the $t^2$ coefficient of $f$ is zero, it follows that $a - 1 = 0$, hence $a = 1$. Thus, we have \[
                f(t) = (t - 1) (t^2 + t - 6) = (t-1) (t+3) (t-2)
            .\] 
            \item Factorising the numerator and denominator, we get \begin{align*}
                \frac{x^3 + x^2 - 2x}{x^3 + 2x^2 - x - 2} &= \frac{x(x - 1)(x + 2)}{(x - 1)(x + 1)(x + 2)} \\
                &= \frac{x}{x+1}.
            \end{align*}
        \end{enumerate}
        \item \textbf{A7} \begin{enumerate}
            \item We have \begin{align*}
                \frac{2}{(x+1)(x-1)} &\equiv \frac{A}{x+1} + \frac{B}{x-1} \\
                &\equiv \frac{(A+B)x + (B - A)}{(x+1)(x-1)}.
            \end{align*}
            Equating coefficients yields the following system for equations \begin{align*}
                A + B &= 0 \\
                B - A &= 2,
            \end{align*}
            which has the following solutions
            \begin{align*}
                A &= -1 \\
                B &= 1.
            \end{align*}
            Substituting these into our original equivalence relation gives \[
                \frac{2}{(x+1)(x-1)} = \frac{1}{x-1}  - \frac{1}{x+1}
            .\] 
            \item Similarly, we have \begin{align*}
                \frac{x+13}{(x+1)(x-2)(x+3)} &\equiv \frac{A}{x+1} + \frac{B}{x-2} + \frac{C}{x+3} \\
                &\equiv \frac{A(x-2)(x+3) + B(x+1)(x+3) + C(x+1)(x-2)}{(x+1)(x-2)(x+3)} \\
                &\equiv \frac{(A+B+C)x^2 + (A+4B-C)x+(3B-6A-2C)}{(x+1)(x-2)(x+3)}.
            \end{align*}
            Equating coefficients gives us the following system of equations \begin{align*}
                A + B + C &= 0 \\
                A + 4B - C &= 1 \\
                3B - 6A - 2C &= 13.
            \end{align*}
            From equation (1) we have \[
                - C = A + B
		.\] Substituting this into equations (2) and (3) yields \begin{align*}
                2A + 5B &= 1 \\
                5B - 4A &= 13.
            \end{align*}
            So the solution is $A = -2, B = 1, C = 1$. And so we finally have \[
                \frac{x+13}{(x+1)(x-2)(x+3)} = \frac{-2}{x+1} + \frac{1}{x-2} + \frac{1}{x+3}
            .\] 
            \item We have \begin{align*}
                \frac{4x+1}{(x+1)^2(x-2)} &\equiv \frac{A}{x-2} + \frac{B}{x+1} + \frac{C}{(x+1)^2} \\
                &\equiv \frac{A(x+1)^2 + B(x-2)(x+1) + C(x-2)}{(x-2)(x+1)^2} \\
                &\equiv \frac{(A + B)x^2 + (2A - B + C)x + (A - 2B - 2C)}{(x-2)(x+1)^2}
            \end{align*}
            Equating coefficients produces the following system of equations \begin{align*}
                A + B &= 0 \\
                2A - B + C &= 4 \\
                A - 2B - 2C &= 1,
            \end{align*}
            From equation (1) we get $A = -B$ hence \begin{align*}
                3A + C &= 4 \\
                3A - 2C &= 1.
            \end{align*}
            So we have the solution $A = 1, B = -1, C = 1$. So the partial fraction decomposition is \[
                \frac{4x+1}{(x+1)^2(x-2)} = \frac{1}{x-2} - \frac{1}{x+1} + \frac{1}{(x+1)^2}
            .\] 
            \item We have \begin{align*}
                \frac{4x^2 + x - 2}{(x-1)(x^2 + 2)} &\equiv \frac{A}{x-1} + \frac{Bx+C}{x^2 + 2} \\
                &\equiv \frac{(A + B)x^2 + (C - B)x + (2A - C)}{(x-1)(x^2 + 2)}.
            \end{align*}
            Equating coefficients gives us the following system of equations \begin{align*}
                A + B &= 4 \\
                C - B &= 1 \\
                2A - C &= -2.
            \end{align*}
            By inspection we see that the solution is $A = 1, B = 3, C = 4$. And so the partial fraction decomposition is \[
                \frac{4x^2 + x - 2}{(x-1)(x^2 + 2)} = \frac{1}{x-1} + \frac{3x + 4}{x^2 + 2}
            .\] 
        \end{enumerate}
        \item \textbf{FC*} See paper. 
        \item \textbf{FC5} \begin{enumerate}
            \item $x = -1/2$.
            \item Recall the definition of a logarithm. $x = k^{\log_k x}$. So if $\log_a b = c$, then it follows that for any base $\alpha$ we have \begin{align*}
                a^c &=  b\\
                \iff \log_\alpha a^c &= \log_\alpha b \\
                \iff c \log_\alpha a &= \log_\alpha b \\
                \iff c &= \frac{\log_\alpha b}{\log_\alpha a}.
            \end{align*}
            As required.
            \item Changing to base $e$ gives us \begin{align*}
                16 \log_x 3 &= 16 \frac{\ln 3}{\ln x} = \frac{\ln x}{\ln 3} = \log_3 x \\
                \iff (\ln x)^2 &= (\ln 3^4)^2 \\
                \iff \ln x_{1,2} &= \pm \ln 3^4 \\
                \iff x_{1,2} &= e^{\pm \ln 3^4}
            \end{align*}
            So the solutions are $x_{1,2} = 81, 1/81$.
        \end{enumerate}
        \item \textbf{G1} \begin{enumerate}
            \item Note that given $AB = BC$ we can deduce that triangle $ABC$ is an isosceles triangle, hence if $\angle A = \pi/3$ then it follows that $\angle B = \pi/3$. Given that \[
                \angle A + \angle B + \angle C = \pi
            .\] Then $\angle C = \pi/3$. Thus $ABC$ is in-fact a equilateral triangle, giving $AB = BC = CA = 1$.
            \item Notice that $ABC$ is an isosceles triangle with $AB = BC = 2$ and a base $AC = 3$. Using a method of dissection we produce two congruent right-angled triangles $ABM$ and $CBM$ where $M$ is the midpoint of $AC$. Let only consider $ABM$ (as the triangles are congruent). Since $ABM$ is a right-angled triangle, we have \[
                \cos \angle A = \frac{3}{4}
            .\]  So $\angle A = \arccos(3/4)$. Since $ABC$ is an isosceles triangle, then $\angle A = \angle B = \arccos(3/4)$. And given that \[
                \angle A + \angle B + \angle C = \pi
            .\] Then $\angle C = \pi - 2\arccos(3/4)$.
        \end{enumerate}
        \item \textbf{G2} \begin{enumerate}
            \item The length $\ell$ of a sector is given by $\ell = r\theta$, where $r$ is the radius of the circle and $\theta$ is the sector angle (in radians). So when $r = 3$ and $\theta = \pi/3$ it follows that $\ell = \pi$.
            \item The area $A$ of a sector is given by $A = \frac{1}{2} r^2 \theta$. So substituting $r= 3$ and $\theta = \pi/3$ gives us $A = 3/2 \pi$.
        \end{enumerate}
        \item \textbf{G3}. We have the lines (in Cartesian form) \begin{align*}
            x = y = z \\
            x = y = 2z + 1.
        \end{align*}
        Writing these lines in a vector form gives us \begin{align*}
            \vec{r}_1 &= \lambda \left\langle 1, 1, 1 \right\rangle \\
            \vec{r}_2 &= \left\langle 0, 0, -1/2 \right\rangle + \mu \left\langle 1, 1, 1/2 \right\rangle.
        \end{align*}
        In order to find the angle between $\vec{r}_1$ and $\vec{r}_2$, let us consider their direction vectors $\vec{b}_1 = \left\langle 1,1,1 \right\rangle$ and $\vec{b}_2 = \left\langle 1,1,1/2 \right\rangle$. Recall that the dot product between to vectors $\vec{v}$ and $\vec{u}$ is $\vec{v} \cdot \vec{u} = \| \vec{v} \| \| \vec{u} \| \cos \theta$, where $\theta$ is the angle between the vectors. Rearranging for $\theta$ gives \[
            \theta = \arccos \left( \frac{\vec{v} \cdot \vec{u}}{\| \vec{v} \| \| \vec{u} \|} \right)
        .\] 
        So we have \[
            \theta = \arccos \left( \frac{5/2}{\sqrt{3} \sqrt{9/4} } \right) = \arccos \left( \frac{5 \sqrt{3}}{9} \right)
        .\] 
        In order to determine whether the lines intersect we need to find $\lambda, \mu$ such that \[
            \lambda \left\langle 1,1,1 \right\rangle = \left\langle 0,0,-1/2 \right\rangle + \mu \left\langle 1,1,1/2 \right\rangle
        .\] which gives us the following system of equations \begin{align*}
            \lambda = \mu \\
            \lambda = -1/2 + 1/2 \mu,
        \end{align*}
        which has the solution $\lambda = \mu = -1$, hence the lines intersect when $\lambda = \mu = -1$.
        \item \textbf{SS1}. Recall that the general term of an arithmetic progression with an initial term $u_1$ and a constant term difference $d$ is given by $u_n = u_1 + (n-1)d$. And so given $\alpha$ is the 3rd term and $\beta$ is the 9th term, we can then form the following system of equations \begin{align*}
            \alpha = u_1 + 2d \\
            \beta = u_1 + 8d.
        \end{align*}
        Rearranging for $d$ yields \[
            d = \frac{\beta - \alpha}{6}
        .\] Substituting this into equation (1) produces \[
            u_1 = \frac{4 \alpha - \beta}{3}
        .\] Also recall that the sum of the first $n$ terms of the arithmetic progression $u_1, u_2, \ldots, u_n$ is \[
            S_n = \sum_{r=1}^{n} u_r = \frac{1}{2} n (2u_1 + (n-1)d)  
        .\] So for the first thirty terms we have \begin{align*}
            S_{30} &= \frac{1}{2} (30) \left( 2 \frac{4 \alpha - \beta}{3} + 29 \frac{\beta - \alpha}{6} \right) \\
            &= \frac{5}{2} \left( 25 \beta -13 \alpha \right).
        \end{align*}
	\item \textbf{SS2} \begin{enumerate} 
            \item Applying the binomial theorem gives us \begin{align*}
                (1 + x)^3 &= \sum_{r = 0}^{3} {3 \choose r} (1)^{3 - r} (x)^{r} \\
                &= 1 + 3x + 3x^2 + x^3.
            \end{align*}
            \item Applying the binomial theorem gives us \begin{align*}
                (2 + x)^4 &= \sum_{r = 0}^{4} {4 \choose r} (2)^{4 - r} (x)^{r} \\
                &= 16 + 32 x + 24 x^2 + 8 x^3 + x^4.
            \end{align*}
            \item We have \begin{align*}
                \left( 2 + \frac{3}{x} \right) ^5 &= \left( \frac{3 + 2x}{x} \right)^5 \\
                &= \frac{1}{x^5} (3 + 2x)^5 
            \end{align*}
            Applying the binomial theorem yields \begin{align*}
                \frac{1}{x^5} (3 + 2x)^5 &= \frac{1}{x^5} \sum_{r = 0}^{5} {5 \choose r} (3)^{5 - r} (2x)^{r} \\
                &= \frac{1}{x^5} \left( 243 + 810x + 1080 x^2 + 720 x^3 + 240 x^4 + 32 x^5 \right)
            \end{align*}
        \end{enumerate}
        \item \textbf{SS3} \begin{enumerate}
            \item Recall that the sum of the first $n$ terms of a arithmetic progression with general term $u_n = a + (n-1)d$ is \[
                S_n = \sum_{r=1}^{n} u_r = \frac{1}{2} n (2a + (n-1) d)
            .\] 
            Given the arithmetic progression $u_n = n$, it follows that \\ $a = 1, d = 1$. So we have \begin{align*}
                \sum_{r=1}^{n} u_r = \frac{1}{2} n (2 + n - 1) = \frac{1}{2} n (n + 1). 
            \end{align*}
            As required.
            \item Consider the arithmetic progression with general term $u_n = 2n + 1$. So the sum of the first $n$ terms of $u_n$ is \[
                S_n = \sum_{r=1}^{n} (2r + 1) = 2 \sum_{r=1}^{n} r + \sum_{r=1}^{n} 1 = n(n + 2)  
            .\]  Given $u_n$ is a sequence of odd integers and $u_5 = 11$ and $u+49 = 99$. Then we have \begin{align*}
                S_{49} - S_{5} &= \sum_{n=1}^{49} (2n + 1) - \sum_{n=1}^{5} (2n+1) \\ 
                &= 49(49 + 2) - 5(5 + 2) \\
                &= 2469
            \end{align*}
            \item We have \[
                \sum_{n=1}^{5} (3n + 2) = 3 \sum_{n=1}^{5} n + 10 = \frac{3}{2} (5)(6) + 10 = 55
            .\] 
            \item We have \begin{align*}
                \sum_{n=0}^{N} (an+b) &= a \sum_{n=1}^{N} n + (N+1)b \\
                &= \frac{a}{2} N (N + 1) + (N+1)b \\
                &= (N+1) \left( \frac{aN}{2} + b \right) 
            \end{align*}
            \item  Recall that the sum of the first $n$ terms of a geometric progression with general term $u_n = ar^{n -1}$ is \[
                S_n = \sum_{k=1}^{n} ar^{k-1} = a\frac{1-r^{n}}{1-r} = a \frac{r^{n} - 1}{r - 1}
            .\] We have \begin{align*}
                \sum_{n=0}^{10} 2^n = \sum_{n=1}^{11} (1) 2^{n - 1} = \frac{1-2^{11}}{1-2} = 2047.
            \end{align*}
            \item Similarly, we have \begin{align*}
                \sum_{n=9}^{N} ar^{2n} &=  \sum_{n=1}^{N+1} ar^{2(n-1)} \\
                &= a \sum_{n=1}^{N+1} (r^2)^{n-1} \\
                &= a \frac{1-r^{2N+2}}{1-r^2}
            \end{align*}
        \end{enumerate}
        \item \textbf{SS4} Given that $u_{n + 1} = k u_n$, then \[
            u_n = k ( k ( k ( \cdots k u_1))) = \underbrace{k \times k \times \cdots \times k}_{n-times} \times u_1 = k^n
        .\]  So we have the following cases as $n \to \infty$: \begin{enumerate}
            \item $k > 1$. $\lim_{n \to \infty} u_n \to \infty$
            \item $k = 1$. $\lim_{n \to \infty} u_n = 1$.
            \item $0 < k < 1$. $\lim_{n \to \infty} u_n \to 0$
            \item $k = 0$. $u_n$ is zero for all $n \neq 0$.
            \item $-1 < k < 0$. $\lim_{n \to \infty} u_n \to 0$, but oscillates about the line $x = 0$.
            \item $k = -1$. $\lim_{n \to \infty} u_n$ oscillates from 1 to $-1$.
            \item $k < -1$. $\lim_{n \to \infty} |u_n| \to \infty$, but oscillates about the line $x = 0$.
        \end{enumerate}
        \item \textbf{SS5} Recall that the Maclaurin series for $(1+x)^n$ is \[
            (1+x)^n = \sum_{r = 0}^ \infty \frac{n(n-1) \cdots (n-r+1)}{r!} x^r
        .\] Which is valid for $|x| < 1$. Using this, we have: \begin{enumerate}
            \item \[
                (1+x)^{\frac{1}{2}} = 1 + \frac{1}{2}x - \frac{1}{8}x^2 + \frac{1}{8}x^3 + O(x^4)
            .\]  This expansion is valid for $|x| < 1$.
            \item \begin{align*}
                (2 + x)^{\frac{2}{5}} &= \sqrt[5]{4} \left( 1 + \frac{x}{2} \right) ^{\frac{2}{5}} \\
                &= \sqrt[5]{4} \left( 1 + \frac{1}{5}x - \frac{3}{100}x^2 + \frac{2}{125}x^3 + O(x^4) \right)
            \end{align*} This expansion is valid for $|x| < 2$
            \item \begin{align*}
                (1+2x)^{\frac{1}{2}} (2+x)^{-\frac{1}{3}} &= 2^{-\frac{1}{3}} \left( 1 + x - \frac{1}{2}x^2 + x^3 + O(x^4) \right) \\ 
                &\times \left( 1 - \frac{1}{6}x + \frac{1}{18}x^2 - \frac{7}{162}x^3 + O(x^4)  \right)
            \end{align*}
            This expansion is valid if and only if $|2x| < 1$ and $\left| \frac{x}{2} \right| < 1$, so the expansion is valid if and only if $|x| < \frac{1}{2}$.
        \end{enumerate}
        \item \textbf{SS6} Using the approximations \begin{align*}
            \sin \theta &\approx \theta - \frac{1}{6}\theta^3 \\
            \cos \theta &\approx 1 - \frac{1}{2}\theta^2,
        \end{align*}
        we have \begin{align*}
            \sin \left( \frac{\theta}{2} \right) \cos \theta + \sec 2\theta &\approx \left( \frac{\theta}{2} - \frac{1}{48} \theta^3 \right) \left( 1 - \frac{1}{2}\theta^2 \right) + \frac{1}{1-2\theta^2} \\
            &= \frac{\theta}{2} \left( 1- \frac{13}{24} \theta^2 + \frac{\theta^4}{48} \right) + \frac{1}{1-2\theta^2}.
        \end{align*}
        Recall the Maclaurin series for $1/(1-x)$ is \[
            \frac{1}{1-x} = \sum_{k=0}^{\infty} x^k 
        .\] Hence \[
            \frac{1}{1-2\theta^2} = 1 + 2\theta^2 + 4 \theta^4 + \cdots
        .\] for all $\theta \in \mathbb{R}$. And so substituting this in yields \begin{align*}
            \sin \left( \frac{\theta}{2} \right) \cos \theta + \sec 2\theta &\approx \frac{\theta}{2} - \frac{13}{48} \theta^3 + 1 + 2\theta^2 \\
            &= 1 + \frac{\theta}{2} + 2\theta^2 - \frac{13}{48}\theta^3
        \end{align*}
        \item \textbf{T1} We have \begin{align*}
            2\sin^2 \theta &= 1 \\
            \iff \sin \theta &= \pm \frac{1}{\sqrt{2}}
        \end{align*}
        Using the CAST mnomic, we get the following general solution \[
            \theta \in \left\{ 2n\pi \pm \frac{\pi}{4} : n \in \mathbb{Z} \right \} \cup \left\{(2n+1)\pi \pm \frac{\pi}{4} : n \in \mathbb{Z}\right\} 
        .\] Given we're only interested in solutions in the domain $[0, 2\pi]$, we then have \[
            \theta_{1, \ldots, 4} = \frac{\pi}{4}, \frac{3\pi}{4}, \frac{5\pi}{4}, \frac{7\pi}{4}
        .\] 
        \item \textbf{T2} We have \begin{align*}
            \frac{\cot^2 x + \sin^2 x}{\cos x + \cosec x} &\equiv \frac{\cosec^2 x - (1 - \sin^2 x)}{\cos x + \cosec x} \\
            &\equiv \frac{\cosec^2 x - \cos^2 x}{\cosec x + \cos x} \\
            &\equiv \frac{(\cosec x - \cos x)(\cosec x + \cos x)}{\cosec x + \cos x} \\
            &\equiv \cosec x - \cos x.
        \end{align*} 
        \item \textbf{T3} Recall that the compound angle formulae are \begin{align*}
            \sin (\phi \pm \psi) &= \sin \phi \cos \psi \pm \cos \phi \sin \psi \\
            \cos (\phi \pm \psi) &= \cos \phi \cos \psi \mp \sin \phi \sin \psi.
        \end{align*} Applying these to the following questions gives: \begin{enumerate}
            \item \begin{align*}
                \cos \pi/12 &= \cos (\pi/3 - \pi/4) \\
                &= \cos \pi/3 \cos \pi/4 + \sin \pi/3 \sin \pi/4 \\
                &= \frac{1}{2} \frac{1}{\sqrt{2} } + \frac{\sqrt{3} }{2} \frac{1}{\sqrt{2} } \\
                &= \frac{\sqrt{6} + \sqrt{2}}{4}.
            \end{align*}
            \item \begin{align*}
                \sin \pi/12 &= \sin (\pi/3 - \pi/4) \\
                &= \sin \pi/3 \cos \pi/4 - \cos \pi/3 \sin \pi/4 \\
                &= \frac{\sqrt{3}}{2} \frac{1}{\sqrt{2}} - \frac{1}{2} \frac{1}{\sqrt{2}} \\
                &= \frac{\sqrt{6} - \sqrt{2}}{4}.
            \end{align*}
            \item \begin{align*}
                \cot \pi/12 &= \frac{\cos \pi/12}{\sin \pi/12} \\
                &= \frac{\sqrt{6} + \sqrt{2}}{\sqrt{6} - \sqrt{2}} \\
                &= 2 + \sqrt{3} .
            \end{align*}
        \end{enumerate}
        \item \textbf{T4} If $t = \tan \frac{\theta}{2}$, then we have a right-angled triangle with opposite $t$, adjacent $1$ and hypotenuse $\sqrt{1+t^2}$. So it follows that \begin{align*}
            \sin \frac{\theta}{2} &= \frac{t}{\sqrt{1 + t^2}} \\
            \cos \frac{\theta}{2} &= \frac{1}{\sqrt{1 + t^2}}.
        \end{align*} Applying the double angle formulae yields \begin{align*}
            \cos \theta &= \frac{1}{1+t^2} - \frac{t^2}{1+t^2} = \frac{1-t^2}{1+t^2}\\
            \sin \theta &= 2 \frac{t}{\sqrt{1 + t^2}} \frac{1}{\sqrt{1+t^2}} = \frac{2t}{1+t^2}.
        \end{align*} And from definition of $\tan$ we get \[
            \tan \theta = \frac{\sin \theta}{\cos \theta} = \frac{2t}{1-t^2}
        .\] 

        \item \textbf{T5} Recall that the compound angle formula for $\tan$ is \[
            \tan (\phi \pm \psi) = \frac{\tan \phi \pm \tan \psi}{1 \mp \tan \phi \tan \psi}
        .\]  So applying this produces \begin{align*}
            \tan \left(\arctan \frac{1}{3} + \arctan \frac{1}{4}\right) &= \frac{\tan \arctan \frac{1}{3} + \tan \arctan \frac{1}{4}}{1 - \tan \arctan \frac{1}{3} \tan \arctan \frac{1}{4}} \\
            &= \frac{\frac{1}{3} + \frac{1}{4}}{1 - \frac{1}{3} \frac{1}{4}} \\
            &= \frac{7}{11}.
        \end{align*}
        \item \textbf{T6} Let us consider any two angles $\phi, \psi$. We wish to know what the product of two sines is. So applying the compound angle formula for $\cos$ we get \begin{align*}
            \cos(\phi - \psi) - \cos (\phi + \psi) &= \left( \cos \phi \cos \psi + \sin \phi \sin \psi \right) - \left( \cos \phi \cos \psi - \sin \phi \sin \psi \right) \\
            &= 2 \sin \phi \sin \psi.
        \end{align*}
        Substituting $A, B$ and $C$ in produces \[
            \cos \left( \frac{B - C}{2} \right) - \cos \left( \frac{B + C}{2} \right) = 2 \sin \frac{B}{2} \sin \frac{C}{2}
        .\] Now since $A,B, C$ are angles of a triangle, it follows that \[
            A + B + C = \pi
        .\] So $B + C = \pi - A$ giving us \[
            \cos \left( \frac{B - C}{2} \right) - \cos \left( \frac{\pi - A}{2} \right) = 2 \sin \frac{B}{2} \sin \frac{C}{2}
        .\] Recall the identity \[
            \cos \left(\frac{\pi}{2} - \theta\right) \equiv \sin \theta
        .\] Thus \[
            \cos \left( \frac{B - C}{2} \right) - \sin \frac{A}{2} = 2 \sin \frac{B}{2} \sin \frac{C}{2}
        .\] 
        As required.
        \item \textbf{T7} We have \begin{align*}
            \sqrt{3}\sin \theta + \cos \theta &\equiv A \sin(\theta + \alpha) = A \sin \theta \cos \alpha + A \cos \theta \sin \alpha.
        \end{align*} Equating coefficients produces the following system of equations \begin{align*}
            \sqrt{3} &= A \cos \alpha \\
            1 &= A \sin \alpha
        \end{align*} Squaring equations (1) and (2) and adding them together gives us \[
            1 + 3 = A^2 (\cos^2 \alpha + \sin^2 \alpha) = A^2
        .\] which implies $A = 2$. Now dividing equations (1) and (2)  \[
            \tan \alpha = \frac{1}{\sqrt{3}}
        .\] Hence $\alpha = \pi/6$.
        \item \textbf{T8}  Recall the triple angle formulae are \begin{align*}
            \cos 3 \theta = 4 \cos^3 \theta - 3 \cos \theta \\
            \sin 3 \theta = 3 \sin \theta - 4 \sin ^3 \theta
        \end{align*} And so we have \begin{align*}
            \cos \theta + \cos 3 \theta &= \sin \theta + \sin 3 \theta \\
            \iff \cos \theta + 4 \cos^3 \theta - 3\cos \theta &= \sin \theta + 3 \sin \theta - 4 \sin^3 \theta \\
            \iff 4 \cos^3 \theta - 2 \cos \theta &= 4 \sin \theta - 4 \sin^3 \theta \\
            \iff \cos^3 \theta (4 - 2 \sec^2 \theta) &= \cos^3 \theta ( 4 \tan \theta \sec^2 \theta - 4 \tan^3 \theta) \\
            \iff \cos^3 \theta (2 - 2\tan^2 \theta) &= \cos^3 \theta (4 \tan \theta) \\
            \iff \cos^3 (\tan^2 \theta + 2 \tan \theta - 1) &= 0
        \end{align*}
        So we have a quadratic equation in terms of $\tan \theta$ or $\cos^3 \theta = 0$. Solving the quadratic equation gives us \[
            \tan \theta = - 1 \pm \sqrt{2}
        .\] and \[
            \cos^3 \theta = 0 \iff \cos \theta = 0
        .\]  So in the domain $[0, 2\pi]$, the solutions are \[
            \theta \in \left\{\frac{\pi}{8}, \frac{5\pi}{8}, \frac{9\pi}{8}, \frac{13\pi}{8}, \frac{\pi}{2}, \frac{3\pi}{2}\right\} 
        .\] 
        \item \textbf{V1} \begin{enumerate}
            \item $\| \vec{A}\| = \sqrt{293}$, $\| \vec{B}\| = \sqrt{293}$, $\| \vec{C}\| = \sqrt{290}$ and $\| \vec{D}\| = 17$, hence $\| \vec{A} \| = \| \vec{B} \| > \| \vec{C} \| > \| \vec{D}$.
            \item Recall that the dot product between to vectors $\vec{v}$ and $\vec{u}$ is $\vec{v} \cdot \vec{u} = \| \vec{v} \| \| \vec{u} \| \cos \theta$, where $\theta$ is the angle between the vectors. Rearranging for $\theta$ gives \[
                \theta = \arccos \left( \frac{\vec{v} \cdot \vec{u}}{\| \vec{v} \| \| \vec{u} \|} \right)
            .\]  Applying the above, we get \begin{enumerate}
                \item The dot product between $\vec{A}$ and $\vec{B}$ is $\vec{A} \cdot \vec{B} = -29$. So \[
                    \theta = \arccos \left( \frac{-29}{293} \right)
                .\] 
                \item The dot product between $\vec{B}$ and $\vec{C}$ is $\vec{B} \cdot \vec{C} = 1$. So \[
                    \theta = \arccos \left( \frac{2}{\sqrt{293} \sqrt{290} } \right)
                .\] 
            \end{enumerate}
        \end{enumerate}
	\item \textbf{V2} 
		\begin{enumerate}
			\item Let us consider the vector $\overrightarrow{AB} = \vec{B} - \vec{A}$. So \[
				\overrightarrow{AB} = \left\langle -12, 20, -10 \right\rangle
			.\] Given that the distance $d$ between the points is equal to the magnitude of the vector $\overrightarrow{AB}$, then if follows that the distance $d$ is \[
				d = \sqrt{12^2 + 20^2 + 10^2} = 2 \sqrt{161}
			.\]
		\end{enumerate}
        
        \item \textbf{D1} \begin{enumerate}
            \item Computing the first and second derivatives of $y$ gives us \begin{align*}
                \frac{\mathop{\mathrm{d}y}}{\mathop{\mathrm{d}x}} &= 2x \\
                \frac{\mathop{\mathrm{d}^{2}y}}{\mathop{\mathrm{d}x^{2}}} &= 2
            \end{align*} Hence we have a minima at $x = 0$ with no points of inflection.
            \item Computing the first and second derivatives of $y$ gives us \begin{align*}
                \frac{\mathop{\mathrm{d}y}}{\mathop{\mathrm{d}x}} &= 3(x^2 - 1) = 3(x+1)(x-1) \\
                \frac{\mathop{\mathrm{d}^{2}y}}{\mathop{\mathrm{d}x^{2}}} &= 6x.
            \end{align*} Hence we have a point of inflection at $x = 0$ and a minima at $x = 1$ and a maximum at $x=-1$.
            \item Computing the first and second derivatives of $y$ gives us \begin{align*}
                \frac{\mathop{\mathrm{d}y}}{\mathop{\mathrm{d}x}} &= 3x^2 - 6x + 3 = 3(x^2 - 2x + 1) = 3(x - 1)^2 \\
                \frac{\mathop{\mathrm{d}^{2}y}}{\mathop{\mathrm{d}x^{2}}}  &= 6(x - 1).
            \end{align*}
            Hence we have a point of inflection at $x = 1$ and a stationary point at $x = 1$.
            \item Computing the first and second derivatives of $y$ gives us \begin{align*}
                \frac{\mathop{\mathrm{d}y}}{\mathop{\mathrm{d}x}} &= 3x^2 + 3 = 3(x^2+1) \\
                \frac{\mathop{\mathrm{d}^{2}y}}{\mathop{\mathrm{d}x^{2}}} &= 6x.
            \end{align*} So we have a point of inflection at $x = 0$. 
        \end{enumerate}
        \item \textbf{D2} Let $f(x) = x^2 + 1$, then by first principles we have \begin{align*}
            f'(x) &= \lim_{\Delta x \to 0} \frac{f(x + \Delta x) - f(x)}{\Delta x} \\ 
            &= \lim_{\Delta x \to 0} \frac{x^2 + 2x \Delta x + \Delta x^2 + 1 - x^2 - 1}{\Delta x} \\
            &= \lim_{\Delta x \to 0} \frac{2x \Delta x + \Delta x^2}{\Delta x} \\
            &= \lim_{\Delta x \to 0} 2x + \Delta x \\
            &= 2x.
        \end{align*} 
        \item \textbf{D3} \begin{enumerate}
            \item \[
                \frac{\mathop{\mathrm{d}y}}{\mathop{\mathrm{d}x}} = 2x \sin x^2 
            .\] 
            \item \[
                \frac{\mathop{\mathrm{d}y}}{\mathop{\mathrm{d}x}} = a^x \ln x
            .\] 
            \item Note that \[
                y = \ln (2x^a + 1) - \ln x^a = \ln (2x^a + 1) - a\ln x
            .\] So we have \[
                \frac{\mathop{\mathrm{d}y}}{\mathop{\mathrm{d}x}} = \frac{2ax^{a-1}}{2x^a + 1} - \frac{a}{x}
            .\] 
            \item By definition of logarithms, we have \[
                y = x^x = e^{\ln x^x} = e^{x \ln x}
            .\] So \[
                \frac{\mathop{\mathrm{d}y}}{\mathop{\mathrm{d}x}} = e^{x \ln x} \times (\ln x + 1) = x^x (\ln x + 1)
            .\] 
            \item We have \begin{align*}
                y &= \arcsin x \\
                \iff \sin y &= x \\
                \iff \frac{\mathop{\mathrm{d}}}{\mathop{\mathrm{d}x}} \sin y &= \frac{\mathop{\mathrm{d}}}{\mathop{\mathrm{d}x}} x \\
                \iff \cos y \frac{\mathop{\mathrm{d}y}}{\mathop{\mathrm{d}x}} &= 1 \\
                \iff \frac{\mathop{\mathrm{d}y}}{\mathop{\mathrm{d}x}} &= \frac{1}{\cos y}
            \end{align*}
            Given $\rng(\arcsin)  = [-\pi/2, \pi/2]$ so it follows that $\cos y$ is positive for all $y \in [-\pi/2, \pi/2]$. Using the Pythagorean identity, we have \[
                \cos y = \sqrt{1 - x^2} 
            .\] Hence \[
                \frac{\mathop{\mathrm{d}y}}{\mathop{\mathrm{d}x}} = \frac{1}{1-x^2}
            .\] 
        \end{enumerate}
        \item \textbf{D4} \begin{align*}
            y + e^y &= x^3 + x + 1 \\
            \iff \frac{\mathop{\mathrm{d}}}{\mathop{\mathrm{d}x}} y + e^y &= \frac{\mathop{\mathrm{d}}}{\mathop{\mathrm{d}x}} x^3 + x + 1 \\
            \iff \frac{\mathop{\mathrm{d}y}}{\mathop{\mathrm{d}x}} (1 + e^y) &= 3x^2 + 1 \\
            \iff \frac{\mathop{\mathrm{d}y}}{\mathop{\mathrm{d}x}} &= \frac{3x^2+1}{1+e^y}
        \end{align*}
        \item \textbf{D5} We have \begin{align*}
            \frac{\mathop{\mathrm{d}y}}{\mathop{\mathrm{d}t}} &= \frac{(t-2) - (t+1)}{(t-2)^2} = -\frac{3}{(t-2)^2}\\
            \frac{\mathop{\mathrm{d}x}}{\mathop{\mathrm{d}t}} &= \frac{2(t-3) - (2t+1)}{(t-3)^2} = - \frac{7}{(t-3)^2}
        \end{align*} Applying the chain rule yields \[
            \frac{\mathop{\mathrm{d}y}}{\mathop{\mathrm{d}x}} = \frac{3}{(t-2)^2} \frac{(t-3)^2}{7}
        .\] Hence \[
            \frac{\mathop{\mathrm{d}y}}{\mathop{\mathrm{d}x}} \Big|_{t = 1} = \frac{12}{7}
        .\] 
        \item \textbf{I1} \begin{enumerate}
            \item Let $x = \sqrt{2} \tan \theta$, then $\mathop{\mathrm{d}x} = \sqrt{2} \sec^2 \theta \mathop{\mathrm{d}\theta}$. Hence \begin{align*}
                \int_{}^{} \frac{1}{2+x^2} \mathop{\mathrm{d}x} &= \int_{}^{} \frac{1}{2 + 2 \tan^2 \theta}   \sqrt{2} \sec^2 \theta \mathop{\mathrm{d}\theta} \\
                &= \frac{1}{\sqrt{2}} \int_{}^{} \frac{\sec^2 \theta}{1 + \tan^2 \theta} \mathop{\mathrm{d}\theta}  \\
                &=  \frac{1}{\sqrt{2}} \int_{}^{} \mathop{\mathrm{d}\theta} \\
                &= \frac{1}{\sqrt{2}} \theta + \kappa 
            \end{align*}
            Thus \[
                \int_{}^{} \frac{1}{2+x^2} \mathop{\mathrm{d}x} = \frac{1}{\sqrt{2}} \arctan \left( \frac{x}{\sqrt{2}} \right)  + \kappa
            .\] 
            \item Let $x - 1 = 2 \sin \theta$, then $\mathop{\mathrm{d}x} = 2 \cos \theta \mathop{\mathrm{d}\theta} $ Thus \begin{align*}
                \int_{}^{} \frac{1}{\sqrt{3 + 2x - x^2}} \mathop{\mathrm{d}x}  &= \int_{}^{} \frac{1}{\sqrt{4 - (x-1)^2} } \mathop{\mathrm{d}x} \\
                &= \int_{}^{} \frac{1}{\sqrt{4 - 4 \sin^2 \theta} } 2 \cos \theta \mathop{\mathrm{d}\theta} \\
                &= \frac{1}{2} \int_{}^{} \frac{\cos \theta}{\cos \theta} \mathop{\mathrm{d}\theta} \\
                &= \frac{1}{2} \theta + \kappa
            \end{align*}
            Hence \[
                \int_{}^{} \frac{1}{\sqrt{3 + 2x - x^2}} \mathop{\mathrm{d}x} = \frac{1}{2} \arcsin \left( \frac{x-1}{2} \right) + \kappa
            .\] 
            \item Let $u = \sqrt{1-x}$, then $\mathop{\mathrm{d}u} = - \frac{1}{2\sqrt{1-x}}$. So we have \begin{align*}
                I = \int_{}^{} \frac{1}{x \sqrt{1-x} } \mathop{\mathrm{d}x} &= - \int_{}^{} \frac{2 \mathop{\mathrm{d}x} }{1 - u^2} \\
                &= \int_{}^{} \frac{2 \mathop{\mathrm{d}u}}{(u + 1) (u - 1)}    
            \end{align*} 
            Notice that the integrand can be expressed as a partial fraction, hence \begin{align*}
                \frac{2}{(u + 1) (u - 1)} &\equiv \frac{A}{u + 1} + \frac{B}{u - 1}\\
                &\equiv \frac{(A + B) u + (B - A)}{(u+1)(u-1)}
            \end{align*}
            Equating coefficients produces the following system of equations \begin{align*}
                A + B &= 0 \\
                B - A &= 2
            \end{align*}
            So by inspection we have the solution $A, B = -1, 1$. Substituting this back into our integrand we get \begin{align*}
                I &= \int_{}^{} \frac{1}{u-1} \mathop{\mathrm{d}u} - \int_{}^{} \frac{1}{u + 1} \mathop{\mathrm{d}u} \\
                &= \ln \left| u - 1 \right| - \ln \left| u + 1 \right| \\
                &= \ln \left| \sqrt{1-x} -1 \right| - \ln \left| \sqrt{1-x} +1 \right| + \kappa \\
                &= \ln \left| \frac{\sqrt{1-x} -1}{\sqrt{1-x} +1} \right| + \kappa
            \end{align*}
            \item Recall the integration by parts states \[
                \int_{}^{} u \frac{\mathop{\mathrm{d}v}}{\mathop{\mathrm{d}x}} \mathop{\mathrm{d}x} = uv - \int_{}^{} v \frac{\mathop{\mathrm{d}u}}{\mathop{\mathrm{d}x}} \mathop{\mathrm{d}x}  
            .\] Now let \[
                u = \ln x \text{ and } \frac{\mathop{\mathrm{d}v}}{\mathop{\mathrm{d}x}} = 1
            .\] Giving us \[
                \frac{\mathop{\mathrm{d}u}}{\mathop{\mathrm{d}x}} = \frac{1}{x} \text{ and } v = x
            .\] Applying the rule we state above yields \[
                \int_{}^{} \ln x \mathop{\mathrm{d}x} = x \ln x - \int_{}^{} \frac{x}{x} \mathop{\mathrm{d}x} = x \ln x - x + \kappa  
            .\] 
        \end{enumerate}
        \item \textbf{I2} \begin{enumerate}
            \item We'll apply integration by parts again using \[
                u = x \text{ and } \frac{\mathop{\mathrm{d}v}}{\mathop{\mathrm{d}x}} = e^{-x}
            .\] then \[
                \frac{\mathop{\mathrm{d}u}}{\mathop{\mathrm{d}x}} = 1 \text{ and } v = - e^{-x}
            .\] So we have \begin{align*}
                \int_{0}^{L} xe^{-x} \mathop{\mathrm{d}x} &= \left[ -xe^{-x} \right]_0^L + \int_{0}^{L} e^{-x} \mathop{\mathrm{d}x} \\
                &= -L e^{-L} + \left[  - e^{-x} \right]_0^L \\
                &= -L e^{-L} + \left( -e^{-L} + 1 \right) \\
                &= 1 - e^{-L} (L - 1)
            \end{align*}
            Now let us consider the limiting value of the integral as $L \to \infty$. So we have \begin{align*}
                \lim_{L \to \infty} \int_{0}^{L} xe^{-x} \mathop{\mathrm{d}x} &= \lim_{L \to \infty} 1 - \frac{L - 1}{e^L} 
            \end{align*}
            On the right hand side, we have \[
                1 - \frac{\infty}{\infty}
            .\] Thus we must apply L'Hopital's rule. So \begin{align*}
                \lim_{L \to \infty} \frac{L - 1}{e^L} &= \lim_{L \to \infty} \frac{1}{e^L} \\
                &\to 0^+
            \end{align*}
            Therefore \[
                \lim_{L \to \infty} \int_{0}^{L} xe^{-x} \mathop{\mathrm{d}x} = 1 
            .\] 
            \item Recall that the triple angle formula for $\sin 3 \theta$ is \[
                \sin 3\theta = 3 \sin \theta - 4 \sin^3 \theta
            .\] Substituting this into the integrand yields \begin{align*}
                I = \int_{0}^{\pi/2} \sin 3\theta \cos \theta \mathop{\mathrm{d}\theta} &= \int_{0}^{\pi/2} (3 \sin \theta - 4 \sin^3 \theta) \cos \theta \mathop{\mathrm{d}\theta}   \\
                &= \int_{0}^{\pi/2} 3\sin \theta \cos \theta \mathop{\mathrm{d}\theta} - \int_{0}^{\pi/2} 4\sin^3 \theta \cos \theta \mathop{\mathrm{d}\theta}   
            \end{align*}
            Using the substitution $u = \sin \theta$ in both integrals gives us \newline $\mathop{\mathrm{d}u} = \cos \theta \mathop{\mathrm{d}\theta}$ and \begin{align*}
                I &= \int_{0}^{1} 3u \mathop{\mathrm{d}u} - \int_{0}^{1} 4u^3 \mathop{\mathrm{d}u} \\
                &= \left[ \frac{3}{2} u^2 \right]_0^1 - \left[ u^4 \right]_0^1 \\
                &= \frac{3}{2} - 1 = \frac{1}{2}      
            \end{align*}
            \item Let $u = x^3 + 3x + 2$, then $\mathop{\mathrm{d}u} = 3(x^2 + 1) \mathop{\mathrm{d}u}$. So \begin{align*}
                \int_{0}^{1} \frac{x^2 + 1}{x^3 + 3x + 2} \mathop{\mathrm{d}x} &= \frac{1}{3} \int_{2}^{6} \frac{1}{u} \mathop{\mathrm{d}u} \\
                &= \frac{1}{3} \left[ \ln \left| u \right| \right]_2^6 \\
                &= \frac{1}{3} \ln 3.  
            \end{align*} 
            \item Using the Weierstrass substitution $t = \tan \frac{\theta}{2}$ (and our answers from \textbf{T4}) which are \begin{align*}
                \cos \theta &= \frac{1 - t^2}{1 + t^2} \\
                \sin \theta &= \frac{2t}{1+ t^2}
            \end{align*} we then have \begin{align*}
                \mathop{\mathrm{d}t} &= \frac{1}{2} \sec^2 \frac{\theta}{2} \mathop{\mathrm{d}\theta} \\
                \iff \mathop{\mathrm{d}t} &= \frac{1}{2} (1 + t^2) \mathop{\mathrm{d}\theta} \\
                \iff \mathop{\mathrm{d}\theta} &= \frac{2 \mathop{\mathrm{d}t}}{1 + t^2}
            \end{align*} Hence \begin{align*}
                I &= \int_{0}^{\pi/2} \frac{1}{3+5\cos \theta} \mathop{\mathrm{d}\theta} \\
                &= \int_{0}^{1} \frac{1}{3+5 \frac{1 - t^2}{1 + t^2}} \frac{2 \mathop{\mathrm{d}t} }{1 + t^2} \\
                &= \int_{0}^{1} \frac{1 + t^2}{8 - 2t^2} \frac{2}{1 + t^2} \mathop{\mathrm{d}t} \\
                &= \int_{0}^{1} \frac{\mathop{\mathrm{d}t} }{4-t^2}\\
                &= \int_{0}^{1} \frac{\mathop{\mathrm{d}t}}{(2 - t)(2 + t)}     
            \end{align*}
            Notice that the integrand can be expressed as a partial fraction, thus \begin{align*}
                \frac{1}{(2-t)(2+t)} &\equiv \frac{A}{2-t} + \frac{B}{2+t} \\
                &\equiv \frac{(A - B) t + 2(A + B)}{(2-t)(2+t)}
            \end{align*}
            Equating coefficients produces the following system of equations \begin{align*}
                A - B &= 0 \\
                2 (A + B) &= 1
            \end{align*}
            So by inspection we have the solution $A = \frac{1}{4}$ and $B = - \frac{1}{4}$. Substituting this back into our integrand yields \begin{align*}
                I &= \frac{1}{4} \left\{ \int_{0}^{1} \frac{1}{2 - t} \mathop{\mathrm{d}t} - \int_{0}^{1} \frac{1}{2 + t} \mathop{\mathrm{d}t} \right\}  \\
                &= \frac{1}{4} \left\{ \left[  - \ln | 2 - t | \right]_0^1 - \left[ \ln \left| 2 + t \right| \right]_0^1 \right\} \\
                &= \frac{1}{4} \left\{ \left( - \ln 1 + \ln 2 \right) - \left( \ln 3 - \ln 2 \right) \right\} \\
                &= \frac{1}{4} \ln 3
            \end{align*}
        \end{enumerate}
        \item \textbf{DE1} We have \begin{align*}
            x \frac{\mathop{\mathrm{d}y}}{\mathop{\mathrm{d}x}} + 1 - y^2 &= 0 \\
            \iff x \frac{\mathop{\mathrm{d}y}}{\mathop{\mathrm{d}x}} &= y^2 - 1 \\
            \iff \frac{1}{y^2 - 1} \frac{\mathop{\mathrm{d}y}}{\mathop{\mathrm{d}x}} &= \frac{1}{x}
        \end{align*}
        Integrating both sides with respect to $x$ gives us \begin{align*}
            \int_{}^{} \frac{1}{y^2 - 1} \frac{\mathop{\mathrm{d}y}}{\mathop{\mathrm{d}x}} \mathop{\mathrm{d}x} &= \int_{}^{} \frac{\mathop{\mathrm{d}x} }{x} \\
            \iff \int_{}^{} \frac{\mathop{\mathrm{d}y} }{y^2 - 1} &= \int_{}^{} \frac{\mathop{\mathrm{d}x} }{x}    
        \end{align*}
        Notice that the integrand on the right hand side can be expressed as a partial fraction, hence \begin{align*}
            \frac{1}{(y + 1)(y - 1)} &\equiv \frac{A}{y + 1} + \frac{B}{y - 1} \\
            &\equiv \frac{(A + B)y + (B - A)}{(y + 1) (y - 1)}
        \end{align*}
        Equating coefficient produces the following system of equations \begin{align*}
            A + B &= 0 \\
            B - A &= 1
        \end{align*}
        By inspection, we have the solution $A = -1/2$ and $B = 1/2$. Substituting this back into our original integrand yields \begin{align*}
            \frac{1}{2} \left\{ \int_{}^{} \frac{\mathop{\mathrm{d}y} }{y - 1} - \int_{}^{} \frac{\mathop{\mathrm{d}y} }{y + 1}   \right\}  &= \ln \left| x \right| + \kappa_1 \\
            \iff \frac{1}{2} \left\{ \ln \left| y - 1 \right| - \ln \left| y + 1 \right| \right \} + \kappa_2 &= \ln \left| x \right| + \kappa_1 \\
            \iff \ln \left| \frac{y - 1}{y + 1} \right| &= 2 \ln \left| x \right| + \kappa \\
            \iff \frac{y-1}{y+1} &= \lambda x^2
        \end{align*}
        Given the initial value condition $(x, y) = (1, 0)$ we have \[
            \frac{-1}{1} = \lambda \implies \lambda = -1
        .\] 
        So the particular solution is \begin{align*}
            \frac{y - 1}{y + 1} &= -x^2 \\
            \iff y - 1 &= -x^2 y - x^2 \\
            \iff y(1 + x^2) &= 1 - x^2 \\
            \iff y &= \frac{1 - x^2}{1 + x^2}
        \end{align*}
        \item \textbf{C1} \begin{enumerate}
            \item Let us first express \[
                z = \frac{1 + i}{2 - i}
            .\] in the form $z = a + bi$, where $a, b \in \mathbb{R}$. Rationalising the denominator gives us \begin{align*}
                z &= \frac{1 + i}{2 - i} \frac{2 + i}{2 + i} \\
                &= \frac{1 + 3i}{5}.
            \end{align*}
            And so \begin{align*}
                \Re(z) &= \frac{1}{5} \\
                \Im(z) &= \frac{3}{5}.
            \end{align*}
            \item Applying the quadratic formula gives us \begin{align*}
                z_{1,2} &= \frac{2 \pm \sqrt{4 - 4 \times 1 \times 2} }{2} \\
                &= 1 \pm i
            \end{align*}
            The modulus of each root is \[
                \left| z_{1,2} \right| = \sqrt{1^2 + (\pm 1)^2} = \sqrt{2}  
            .\] The argument of $z_1$ is \[
                \arg z_1 = \pi/4
            .\] By symmetry argument, it follows that the argument of $z_2$ is \[
                \arg z_2 = -\pi / 4
            .\] 
            See paper for Argand diagram.
        \end{enumerate}
        \item \textbf{C2} \begin{enumerate}
            \item Recall that de Moivre's theorem states that \[
                (\cos \theta + i \sin \theta) ^ n = \cos n \theta + i \sin n \theta
            .\] So let us consider consider $z = (\cos \theta + i \sin \theta)^5$, hence by de Moivre's theorem $\Re(z) = \cos 5 \theta$. Applying the binomial theorem to $z$ gives us \begin{align*}
                z = \cos 5 \theta + i \sin 5 \theta &= \sum_{r = 0}^{5} {5 \choose r} (\cos \theta)^{5 - r} (i\sin\theta)^{r} \\
                &= \cos^5 \theta + 5i \cos^4 \theta \sin \theta - 10\cos^3 \theta \sin^2 \theta - 10i \cos^2 \theta \sin^3 \theta \\
                &\hspace{5mm} + 5\cos \theta \sin^4 \theta + i \sin^5 \theta \\
                &= \cos \theta (\cos^4 \theta - 10 \cos^2 \theta \sin^2 \theta + 5 \sin^4 \theta) \\
                &\hspace{5mm} + i \sin \theta (5\cos^4 \theta - 10 \cos^2 \theta \sin^2 \theta + \sin^4 \theta)
            \end{align*} 
            Hence \[
                \Re(z) = \cos 5\theta = \cos \theta (\cos^4 \theta - 10 \cos^2 \theta \sin^2 \theta + 5 \sin^4 \theta)
            .\] 
            \item See paper 
        \end{enumerate}
        \item \textbf{VE1} \begin{enumerate}
            \item The vectors $\vec{A}, \vec{B}, \vec{C}$ lie on the line on the line $\vec{r} = \vec{a} + \lambda \vec{b}$ if and only if $\overrightarrow{AB}$ is parallel to $\overrightarrow{BC}$, that is to say $\overrightarrow{BC} = \lambda \overrightarrow{AB}$, where $\lambda \in \mathbb{R}$. Note that \[
                \overrightarrow{AB} = \vec{B} - \vec{A} = \left\langle 1, 1, -1 \right\rangle 
            .\] and \[
                \overrightarrow{BC} = \vec{C} - \vec{B} = -2\left\langle 1,1, -1 \right\rangle = -2 \overrightarrow{AB}
            .\] 
            Hence $\overrightarrow{AB}$ and $\overrightarrow{BC}$ are parallel. Thus the equation of the line they all lie on must have the directional vector $\vec{b} = \left\langle 1,1,-1 \right\rangle$. So the equation of our line is \[
                \vec{r} = \left\langle 1, 0, 1 \right\rangle + \lambda \left\langle 1,1,-1 \right\rangle
            .\] 
        \end{enumerate}
        \item \textbf{M1} \begin{enumerate}
            \item \begin{align*}
                A + B &= \begin{pmatrix} 
                    1 & 2 \\
                    3 & 4
                \end{pmatrix} + \begin{pmatrix} 
                    -2 & -1 \\
                    4 & 2
                \end{pmatrix} \\
                &= \begin{pmatrix} 
                    -1 & 1 \\
                    7 & 6
                \end{pmatrix}.
            \end{align*}
            \item \begin{align*}
                AB = \begin{pmatrix} 
                    1 & 2 \\
                    3 & 4
                \end{pmatrix} \begin{pmatrix} 
                    -2 & -1 \\
                    4 & 2
                \end{pmatrix} = \begin{pmatrix} 
                    6 & 3 \\
                    10 & 5
                \end{pmatrix}.
            \end{align*}
            \item \begin{align*}
                BA = \begin{pmatrix} 
                    -2 & -1 \\
                    4 & 2
                \end{pmatrix} \begin{pmatrix} 
                    1 & 2 \\
                    3 & 4
                \end{pmatrix} = \begin{pmatrix} 
                    -5 & -8 \\
                    10 & 16
                \end{pmatrix}.
            \end{align*}
        \end{enumerate}
        \item \textbf{M2} Let us consider the matrices \begin{align*}
            A &= \begin{pmatrix} p & 0 \\ 0 & 0 \end{pmatrix} \\
            B &= \begin{pmatrix} 0 & 0 \\ q & r \end{pmatrix},
        \end{align*}
        where $p, q, r \in \mathbb{R}$. So we have \begin{align*}
            AB &= \begin{pmatrix} 
                p & 0 \\ 
                0 & 0 
            \end{pmatrix} \begin{pmatrix} 
                0 & 0 \\
                q & r
            \end{pmatrix} \\
            &= \begin{pmatrix} 
                0 & 0 \\
                0 & 0
            \end{pmatrix}.
        \end{align*}
        However, \begin{align*}
            BA &= \begin{pmatrix} 
                0 & 0 \\
                q & r
            \end{pmatrix} \begin{pmatrix} 
                p & 0 \\ 
                0 & 0 
            \end{pmatrix} \\
            &= \begin{pmatrix} 
                0 & 0 \\
                pq & 0
            \end{pmatrix}
        \end{align*}
        Why does this work (and why pick $A$ and $B$)? It is because $A$ and $B$ are singular. 
        If $AB = 0$, then it implies that at least one of the two matrices is singular. 
        \begin{proof}
            Assume there exists two non-singular matrices $A$ and $B$ such that $AB = 0$. 
            It follows that if they are non-singular, then the matrices $A^{-1}$ and $B^{-1}$ exist such that \begin{align*}
                A^{-1} AB &= B \\
                AB B^{-1} &= A
            \end{align*}
            But $AB = 0$, and since $M \times 0 = 0$, then it follows that $A = B = 0$. A contradiction! 
            Thus proving the initial assumption that $A, B$ are non-singular is false. (\textit{I got bored...})
        \end{proof} 
        \item \textbf{M3} Let $A$ be a $2 \times 2$ matrix such that \[
            A = \begin{pmatrix} 1 & -1 \\ 1 & 1 \end{pmatrix}
        .\] Recall that if $M$ is an $n \times n$ matrix and $\lambda$ is some scalar, then \[
            \det(\lambda M) = \lambda^n \det M
        .\] and an enlargement by a scale factor $\lambda$ on the matrix $N$ is \[
            N' = \lambda N
        .\] Now if $A$ is a transformation of a rotation $R$ followed by an enlargement $S$ of scale factor $\lambda$, then $A = SR = \lambda R$. We also note that $R$ will be in the form \[
            R = \begin{pmatrix} \cos \theta & \sin \theta \\ - \sin \theta & \cos \theta \end{pmatrix}
        .\] Which by inspection has the determinant $\det R = 1$ (due to the Pythagorean identity). Given that \[
            \det A = \begin{vmatrix} 1 & -1 \\ 1 & 1 \end{vmatrix} = 1 \times 1 + 1 \times 1 = 2
        .\]  Then \[
            \det A = 2 = \lambda^2
        .\] Hence $\lambda =  \sqrt{2}$ (We ignore the negative solution as it's not relevant to this question). So \[
            A = \sqrt{2} \begin{pmatrix} \frac{1}{\sqrt{2}} & - \frac{1}{\sqrt{2}} \\  \frac{1}{\sqrt{2}} &  \frac{1}{\sqrt{2}} \end{pmatrix} \equiv  \sqrt{2} \begin{pmatrix} \cos \theta & \sin \theta \\ - \sin \theta & \cos \theta \end{pmatrix}
        .\] Equating elements of the matrices produces the following system of equations \begin{align*}
            \cos \theta &=  \frac{1}{\sqrt{2}} \\
            \sin \theta &= - \frac{1}{\sqrt{2}}
        \end{align*}
        Giving us $\tan \theta = -1$. So the general solution is \[
            \theta \in \left\{ 2n \pi - \frac{\pi}{4} : n \in \mathbb{Z} \right \} \cup \left\{ (2n+1) \pi - \frac{\pi}{4} : n \in \mathbb{Z} \right \}
        .\] For simplicity, we'll use the solution $\theta = 3\pi / 4$. Hence \[
            A = \begin{pmatrix} \sqrt{2} & 0 \\ 0 & \sqrt{2} \end{pmatrix} \begin{pmatrix} \cos \frac{3\pi}{4} & \sin \frac{3\pi}{4} \\ -\sin \frac{3\pi}{4} & \cos \frac{3\pi}{4} \end{pmatrix}
        .\] 
        \item \textbf{SE1} \begin{enumerate}
            \item \[
                \sum_{r=1}^{n} r^2 = \frac{1}{6} n (n + 1) (2n + 1)
            .\] 
            \item Recall that \[
                \sum_{r=1}^{n} r^3 = \frac{1}{4}n^2(n+1)^2 \text{ and } \sum_{r=1}^{n} r = \frac{1}{2}n(n+1) 
            .\] So using the above formulae and the linearity of the summation, we have \begin{align*}
                S_n = \sum_{r=1}^{n} r(r^2 + 2) &= \sum_{r=1}^{n} r^3 + 2 \sum_{r=1}^{n} r \\
                &= \frac{1}{4}n^2 (n + 1)^2 + n (n+1) \\
                &= n(n+1) \left( \frac{1}{4}n(n+1) + 1 \right)  
            \end{align*}
        \end{enumerate}
        \item \textbf{SE2} \begin{enumerate}
            \item Note that the general term of the summation can be expressed as a partial fraction, so \begin{align*}
                \frac{1}{r(r+1)} &\equiv \frac{A}{r} + \frac{B}{r+1} \\
                &\equiv \frac{(A + B)r + A}{r(r+1)}
            \end{align*}
            Equating coefficients produces the following system of equations \begin{align*}
                A + B &= 0 \\
                A &= 1
            \end{align*}
            So the solution is $A = 1, B = -1$. Hence the summation $S_n$ can now be expressed as \[
                S_n = \sum_{r=1}^{n} \frac{1}{r} - \frac{1}{r+1} \equiv \sum_{r=1}^{n} f(r) - f(r-1)
            .\] 
            Note that the right hand side equivalence relation is the general form a telescopic series on which we can apply the method of difference. Hence we can apply the method of difference on $S_n$, giving us \begin{align*}
                S_n &= \frac{1}{1} - \frac{1}{2} \\
                    &+ \frac{1}{2} - \frac{1}{3} \\
                    &\hspace{2mm}\vdots \\
                    &+ \frac{1}{n-1} - \frac{1}{n} \\
                    &+ \frac{1}{n} - \frac{1}{n+1}
            \end{align*}
            Cancelling terms produces \[
                S_n = 1 - \frac{1}{n+1} = \frac{n}{n+1}
            .\] 
        \end{enumerate}
        \item \textbf{IN1} Given that $a_1 = 1$ and $a_{n +1} = 3 a_n + 4$, then we have the sequence \[
            1, 7, 25, 79, \ldots = 1, 3 + 4, 3^2 + 4(3 + 1), 3^3 + 4(3^2 + 3 + 1), \ldots
        .\]  By inspection, we see that \begin{align*}
            a_n = 3^{n - 1} + 4 \sum_{r=1}^{n-1} 3^{r-1}
        \end{align*}
        Recall that \[
            \sum_{k=1}^{n} ar^{k - 1} = a \frac{1 - r^n}{1 - r} = a \frac{r^n - 1}{r - 1}
        .\] Hence \[
            a_n = 3^{n - 1} + 4 \frac{3^{n-1}-1}{2} = 3^{n-1} + 2(3^{n-1} - 1) = 3^n - 2
        .\] We will now prove that this is true for all $n \in \mathbb{Z}^+$ by induction on $n$.
        \begin{proof} \hspace{1mm} \\
            \textbf{Base Case}: When $n = 1$, we have $a_1 = 3^1 - 2 = 1 = a_1$. So the statement holds when $n = 1$. \\
            \textbf{Inductive Hypothesis}: Suppose the statement holds for all integer values of $n$ up to some integer $k$, $k \geq 1$. \\
            \textbf{Inductive Step}: Now let us consider $n = k + 1$. So \begin{align*}
                a_{k+1} &= 3 a_k + 4 \\
                &= 3 \left( 3^k - 2 \right) + 4\\
                &= 3^{k+1} - 6 + 4 \\
                &= 3^{k + 1} - 2.
            \end{align*} 
            So, the statement holds for $n = k + 1$. By the principle of mathematical induction, the statement holds for all $n \in \mathbb{Z}^+$.
        \end{proof}
        \item \textbf{IN2} Let \[
            \Pi(n) = \int_{0}^{\infty} x^n e^{-x} \mathop{\mathrm{d}x}   
        .\] We wish to prove that \[
            \Pi(n) = n!
        ,\] for all $n \in \mathbb{N}$. We will do so by induction on $n$. 
        \begin{proof} \hspace{1mm} \\
            \textbf{Base Case}: When $n = 0$, we have \begin{align*}
                \Pi(0) &= \lim_{L \to \infty}  \int_{0}^{L} e^{-x} \mathop{\mathrm{d}x} \\ 
                &= \lim_{L \to \infty}  \left[ - e^{-x} \right]_0^{L} \\
                &= \lim_{L \to \infty} 1 - e^{-L} \\
                &\to 1 = 0!  
            \end{align*}
            So the statement holds for $n = 0$. \\
            \textbf{Inductive Hypothesis}: Suppose the statement holds for all integer values of $n$ up to some integer $k$, $k \geq 0$. \\
            \textbf{Inductive Step}: Now let us consider $n = k + 1$. So \begin{align*}
                \Pi(k+1) &= \lim_{L \to \infty} \int_{0}^{L} x^{k + 1}e^{-x} \mathop{\mathrm{d}x}  
            \end{align*} Recall that integration by parts states \[
                \int_{}^{} u \frac{\mathop{\mathrm{d}v}}{\mathop{\mathrm{d}x}} \mathop{\mathrm{d}x} = uv - \int_{}^{} v \frac{\mathop{\mathrm{d}u}}{\mathop{\mathrm{d}x}} \mathop{\mathrm{d}x}   
            .\] So let $u = x^{k+1}$ and $\frac{\mathop{\mathrm{d}v}}{\mathop{\mathrm{d}x}} = e^{-x}$, then $\frac{\mathop{\mathrm{d}u}}{\mathop{\mathrm{d}x}} = (k+1) x^k$ and $v = -e^{-x}$. Hence \begin{align*}
                \Pi(k + 1) &= \lim_{L \to \infty} \left[ - x^{k+1}e^{-x} \right]_0^L + (k+1) \int_{0}^{L} x^k e^{-x} \mathop{\mathrm{d}x} \\
                &= \lim_{L \to \infty} - \frac{L^{k+1}}{e^{L}} + (k+1) \Pi(k)
            \end{align*} 
            Note that on the right hand side we have $\infty/\infty$. So we must apply L'Hopital's rule $k+1$ times, giving us \[
                \lim_{L \to \infty} \frac{L^{k + 1}}{e^L} = \lim_{L \to \infty} \frac{(k+1)L^k}{e^L} = \cdots = \lim_{L \to \infty} \frac{(k+1)!}{e^L} \to 0 
            .\] Hence \[
                \Pi(k+1) = (k+1) \Pi(k)
            .\] Applying our inductive hypothesis yields \[
                \Pi(k + 1) = (k+1) k! = (k+1)!
            .\] So the statement holds for $n = k + 1$. So by the principle of mathematical induction, the statement holds for all $n \in \mathbb{N}$.
        \end{proof} 
        \item \textbf{H1} Recall that the definitions of $\sinh$ and $\cosh$ are \begin{align*}
            \sinh x &= \frac{1}{2}(e^x - e^{-x}) \\
            \cosh x &= \frac{1}{2}(e^x + e^{-x})
        \end{align*}
        So for $\cosh^2 x - \sinh^2 x = 1$, we have \begin{align*}
            \cosh^2 x - \sinh^2 x &= \left( \frac{e^x + e^{-x}}{2} \right)^2 - \left( \frac{e^x - e^{-x}}{2} \right)^2 \\
            &= \frac{e^{2x} + 2 + e^{-2x}}{4} - \frac{e^{2x} - 2 + e^{-2x}}{4} \\
            &= \frac{4}{4} \\
            &= 1.
        \end{align*}
        As required. \\
        For $\sinh 2x = 2 \sinh x \cosh x$, we have \begin{align*}
            \sinh 2x &= \frac{e^{2x} - e^{-2x}}{2} \\
            &= \frac{(e^x+e^{-x})(e^x-e^{-x})}{2} \\
            &= 2 \frac{e^x - e^{-x}}{2} \frac{e^x + e^{-x}}{2} \\
            &= 2 \sinh x \cosh x.
        \end{align*}
        As required.
        \item \textbf{H2} Recall that $\tanh$ is defined as \[
            \tanh x = \frac{\sinh x}{\cosh x} = \frac{e^x - e^{-x}}{e^x + e^{-x}} = \frac{e^{2x} - 1}{e^{2x} + 1} 
        .\] Recall that the quotient rule states \[
            \frac{\mathop{\mathrm{d}}}{\mathop{\mathrm{d}x}} \frac{u(x)}{v(x)} = \frac{u'(x) v(x) - u(x)v'(x)}{v(x)^2}
        .\] So we have \begin{align*}
            \frac{\mathop{\mathrm{d}}}{\mathop{\mathrm{d}x}} \tanh x &= \frac{2e^{2x}(e^{2x} + 1) - 2e^{2x}(e^{2x} - 1)}{(e^{2x} + 1)^2} \\
            &= \frac{4e^{2x}}{e^{4x} + 2e^{2x} + 1} \\
            &= \frac{4}{e^{2x} + 2 + e^{-2x}} \\
            &= \frac{4}{(e^x + e^{-x})^2} \\
            &= \frac{1}{\cosh^2 x} \\
            &= \sech^2 x.
        \end{align*}
        As required.
    \end{enumerate}
    

\end{document}
